%\documentclass[final,5p,times,twocolumn,authoryear]{elsarticle}
\documentclass{article}


\usepackage{amssymb}

%% \usepackage{lineno}
%\journal{Land Use Policy}

\begin{document}

%\begin{frontmatter}



\title{Urban sprawl and evolution of accessibility profiles in Chinese cities}

%\author[label1]{Juste Raimbault\corref{cor1}}
%\address[label1]{Center for Advanced Spatial Analysis, University College London, London, United Kingdom}
%\cortext[cor1]{Corresponding author: juste.raimbault@polytechnique.edu}


\date{}

\maketitle


\begin{abstract}

\end{abstract}

% planned abstract
% The development of public transportation networks and associated transit oriented development policies are efficient tools to mitigate urban sprawl and its negative environmental impacts, especially in terms of commuting emissions. We study in this paper the trajectories in terms of sprawl and low carbon transport accessibility of the nine largest Chinese megacities, from 1990 to 2030 with projected transportation networks and populations. More precisely, we combine the Global Human Settlement Layer database with a temporal public transportation network database including planned transportation lines. We study the link between population distribution and accessibility profiles. This allows quantifying and classifying different patterns of urban sprawl. We estimate therein the greenhouse gases emission gains thanks to public transport. Although they exhibit different temporal trajectories, we find that final profiles for population and accessibility unveil a homogenous distribution of accessibility and no strong inequality between radial layers of the studied cities. This shows that urban sprawl in China has less impact than what could be expected without appropriate transport policies.

%\begin{keyword}
%\sep
%\end{keyword}

%\end{frontmatter}

%\linenumbers


% special issue Land Use Policy: https://www.journals.elsevier.com/land-use-policy/call-for-papers/limiting-urban-sprawl-relations-between-spatial-trends

% idea: TOD policies - how sprawl impact limited by public transport development. accessibility public transport vs car.
% Data: subways networks (finish?), population GHSL FUAs (or ChinaCities?)



\section{Introduction}

\subsection{Urban sustainability and public transport accessibility}



\subsection{Urban sprawl and transit-oriented development}



\subsection{Homothetic scaling and accessibility profiles}

The size and spatial structure of cities are tightly related, as it was shown by \cite{lemoy2020evidence} that full population density radial profiles follow a scaling relationship with population. This was coined as ``homethetic scaling'', as all cities appear then to be scaled versions of another. This property can be derived from the Alonso urban monocentric model \cite{delloye2020alonso}. It furthermore holds for a variety of land-uses \cite{lemoy2021radial}.

Travel times within large functional urban areas similarly follow some radial scaling patterns: \cite{mennicken2019internal} show that congestions leads to higher transport times the closer to the center, and provide empirical evidence for homothetic scaling of transport access.



\subsection{Proposed approach}

We propose in this paper to study the interaction between transport and land-use, using the homothetic radial scaling methodology. Comparing population profiles with accessibility profiles 




\section{Methods}

\subsection{Data}



\subsection{Accessibility measures}






\section{Results}





\section{Discussion}





%\bibliographystyle{elsarticle-harv} 
\bibliographystyle{apalike}
\bibliography{biblio}



%% \appendix

%% \section{}
%% \label{}


\end{document}

\endinput
